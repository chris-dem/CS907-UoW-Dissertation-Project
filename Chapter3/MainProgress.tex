%!TEX root =  ../Report.tex

\section{Current findings}                               
\label{sec:findings}
We will present several attempts and efforts we made across several problems and their
parsimonious reductions. First we want to introduce combine Kleene Logic with $\textbf{PPAD}$. 
We define the following series of Problems


\begin{definitionbox}{Kleene Unary Numbers}{kleene-unary}
    \label{def:kleene-unary}
    Given a number $p \in M$, its unary representation can be depicted as a binary number $\{0,1\}^M$ such that: $k \in M \to 1^k \in \bar{M}$.
    Under Kleene algebra, we mainly refer to notations $M_k^n$ where $k,n,M \in \mathbb{N}_0$  such that:
    $$
    M^k_n \triangleq \{1^j\bot^k \mid \forall j \in \mathbb{N}_0 : j + k \leq n\}
    $$
\end{definitionbox}



\begin{definitionbox}{\scn{HF-StrongSperner} problem}{hf-strong-sperner}
    \textbf{Input}: A hazard-free circuit $\lambda: [M]^n \to \{-1, +1\}^n$, where $M \in n^{O(1)}$, that describes a \textit{StrongSperner} labelling,
    as explained in \ref{def:strong-sperner}, where the inputs are represented using Kleene Unary numbers \ref{def:kleene-unary}
    and $\#1(p) = \bar{p}$.\\
    \textbf{Output}: A point $p \in \bigtimes_{i \in [n]} M^1_n \cup M^0_n$ if and only if: $\lambda(p) =  \bot^n$. 
\end{definitionbox}

We will maily refer to a specific subset of these problems, which we will refer to as \scn{HF-DiscreteStrongSperner}. 
The general idea is that we exploit the hazard-free property of the Strong Sperner function 
In the original problem %% TODO
one can observe that, we only need two points to create a panchromatic solution, where $\lambda(p) = \overline{\lambda(p')}$. These can create a lot of duplicate
solutions and therefore can behave unpredictably. We will mainly focus on subproblems of the \scn{HF-StrongSperner} where two cubes do not overlap.
More formally we define


\begin{definitionbox}{\scn{HF-DiscreteStrongSperner} problem}{hf-strong-sperner}
    An \scn{HF-StrongSperner} instance $\lambda$ such that $\forall p \in [M^1]^n$, if $\lambda(p)= \bot^n$, then
    \begin{gather*}
    \forall j,i \in [n], [\phi^b_i(x)]_j \triangleq \begin{cases} 
        x_i &\text{if }i \neq j\\
        b &\text{if }i = j\\
    \end{cases}\\
    \forall i \in [n],b \in\{0,1\},\exists j \in [n]: \forall x, x' \in \textbf{Res}(\phi^b_i(x)): [\lambda(x)]_j = [\lambda(x')]_j
    \end{gather*}
\end{definitionbox}

Lastly we introduce a specific subset of the above solutions where introduce antipodal panchromatic cubes. 
In the specific type of solution said we argue that two opposite corners of a cube contain opposite colourings.
This notion of antipodal cubes is used in the majority of reductions across literature. More formally we say 


\begin{definitionbox}{\textbf{MinRes} min and max resolution functions}{min-res-func}
    We use the \textbf{MinRes}, to denote the smallest resolution of a kleene string. More formally we can use the following equivalent definitions:
    \begin{align*}
    \forall x \in \{0,1,\bot\}^*, j \in [|x|]:  [\textbf{MinRes}(x)]_j
&\triangleq \begin{cases}
    x_i &\text{if } x_i \in \{0,1\} \\
    0 & \text{otherwise}
    \end{cases}
    \end{align*}
    For the maximum resolution we will use the $\textbf{MaxRes}$ function where we instead replace all $\bot$ with $1$ instead.
\end{definitionbox}


\begin{definitionbox}{\scn{HF-AntipodalStrongSperner} problem}{hf-strong-sperner}
    An \scn{HF-DiscreteStrongSperner} instance $\lambda$ such that $\forall p \in [M^1]^n$, if $\lambda(p)= \bot^n$, 
    and denote $\textbf{MinRes}(x) = \hat{x}$ as the \textit{anchor} of the cube, then:
    $$
    \forall b \in \{0,1\}^n, j \in [n]: [\lambda(x + b)]_j = -[\lambda(x + 1^n \oplus b)]_j
    $$
\end{definitionbox}


\begin{claimbox}{Antipodal discrete squares}{antipodal-discrete-square}
    Given a panchromatic solution of a \scn{HF-AntipodalStrongSperner} $p$, it must be the case then that:
    $$
    \forall j \in \{-1, 1\}^n, \exists! x \in \textbf{Res}(p): \lambda(x) =  j
    $$
\end{claimbox}

\begin{proof}
We proof by induction on $n$. Lets assume base case $n = 2$. We can observe that the only satisfying 
cube that is both antipodal and discrete, is some rotation of the following matrix
$$
\begin{pmatrix}
    \mp & + \\
    - & \pm \\
\end{pmatrix}
$$
Where we denote $+ = (+, +), -=(-,-), \pm = (+,-)$ and $\mp = (-,+)$, we can observe indeed that the panchromatic square contains all the colourings.
For the induction step, we can observe that if we want our colouring function to be both discrete and antipodal, then the following must suffice:
given $k, k'$ the two resolutions of dimension $k+1$, we know that $[\lambda(p_{-(n+1)}(k))] = (b, \bot^n)$ for some $b \in \{0,1\}^n$. But that implies
$[\lambda(p_{-(n+1)}(k'))] = (\bar{b}, \bot^n)$, which indicates a discrete antipodal panchromatic solution. Moreover if $p_{-(n+1)}$ cover $2^n$ labels,
then the whole cube covers all $2^{n+1}$ labels, which by pigeonhole principle means that we must cover all possible combinations of colourings.
\end{proof}

All the problems above are clearly in \textbf{PPAD} as all of them reduce to the \textit{StrongSperner} problem, under only 
binary inputs, since we make the assumption that all such circuits are inherently \textit{natural}. For the purposes of our current
research, we demonstrate a parsimonious reduction from \scn{HF-DiscreteStrongSperner} to \scn{PureCircuit}.


\begin{theorembox}{}{hf-discrete-to-pure}
    $$
    \textbf{\#PPAD}(\scn{HF-DiscreteStrongSperner}) \subseteq \textbf{\#PPAD}(\scn{PureCircuit})
    $$
\end{theorembox}


\begin{proof}
    
\end{proof}


% \begin{definitionbox}{\textbf{PPAD}-complete}{hf-strong-sperner}
%     \textbf{Input}: A hazard-free circuit $\lambda: [M]^n \to \{-1, +1\}^n$, where $M \in n^{O(1)}$, that describes a \textit{StrongSperner} labelling,
%     as explained in \ref{def:strong-sperner}. For $p \in M$, we use its unary representation, such that $\bar{p} \in M = p \in \{0,1\}^{\lceil \log_2(M) \rceil}$
%     and $\#1(p) = \bar{p}$.
%     \textbf{Output}: We consider a solution a point $p \in \bigtimes_{i \in [n]} M^1$ if and only if:
%     $\lambda(p) =  \bot^n$. 
% \end{definitionbox}



% \subsection{Attempt correlating parsimoniously PureCircuit with the EndOfLine}

